\section{Motivazione e Finalità}

Attraverso la GUI vs CLI si può spiegare la differenza tra un programma strutturato con un'interfaccia da linea di comando oppure con una grafica. Ognuna delle due ha aspetti positivi e  negativi. Attraverso questa attività didattica lo studente dovrebbe comprendere meglio le differenze tra le due, e imparare a capire quando conviene usare un'interfaccia piuttosto che un'altra.

Un'ottima sintesi di definizione di CLI la si può trovare sul sito di Golang:
\begin{quote}
    
Command line interfaces (CLIs), unlike graphical user interfaces (GUIs), are text-only. Cloud and infrastructure applications are primarily CLI-based due to their easy automation and remote capabilities. 

\end{quote}

\section{Prerequisiti}
Lo studente dovrebbe avere delle basi di programmazione,come la conoscenza della differenza tra input e output e del concetto di libreria software, infine una conoscenza di  base di matematica; questa non è strettamente necessaria all'attività didattica ma per svolgere in maniera più semplice l'esercizio di programmazione proposto. L'insegnante dovrà sincerarsi che gli studenti conoscano il concetto di fattoriale di un numero, ed eventualmente fornire questa informazione. 

\section{Contenuti}
Come strutturare un progetto basilare, sia come CLI sia come GUI, le principali differenze di progettuali, e quando conviene usare una piuttosto che l'altra

\section{Traguardi e obiettivi}
Gli studenti al termine di questa attività didattica dovrebbero apprendere le principali differenze tra un'interfaccia costruita basandosi sulla linea di comando o su un'interfaccia grafica, valutando anche diverse opzioni, più o meno intuitive, per lo sviluppo grafico. L'obiettivo è evidenziare le differenze tra i due approcci, formando gli strumenti per valutare quale dei due sia più efficace per una particolare applicazione. 
Si propongono due esempi:
\begin{itemize}
\item Problema semplice, in cui è evidente il beneficio in termini di rapidità di implementazione di usare una CLI
\item Problema più complesso che evidenzia i vantaggi dell'uso di una GUI (attività unplugged)
\end{itemize}