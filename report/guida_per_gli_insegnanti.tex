\section{Guida per gli insenganti}
Per l'uso del materiale didattico fornito conviene spiegare la consegna e in seguito lasciare gli studenti ragionare su come risolvere l'esercizio. Solo dopo circa venti  minuti confrontare le soluzioni. Essendo quest'attività didattica strutturata affinché sia efficace con un apprendimento attivo si suggerisce di lasciare gli studenti ragionare tra loro e provare a dare sporadici consigli in modo che si sviluppi un ambiente di Socio-costruttivismo

Gli snodi nelle due proposte di esercizio si possono identificare icome segue nei due esercizi proposti. 
Nell'esercizio unplugged:
\begin{itemize}
\item Comprendere come collegare il comando alla singola condotta.
\item Individuare le possibili opzioni per la realizzazione dell'interfaccia, impostandone la struttura
\item Maturare un giudizio motivato su quale possa essere la soluzione più efficace dal punto di vista dell'utente
\end{itemize}

Nell'esercizio di programmazione:
\begin{itemize}
\item Comprendere la regola per il calcolo del fattoriale
\item Sviluppare il codice di base per calcolare il fattoriale
\item Comprendere come strutture una CLI 
\item Sviluppare la CLI che richiami la funzione per il calcolo del fattoriale
\item Comprendere come strutturare una GUI mediante GTK per Go
\item Sviluppare la GUI che permetta di inserire il numero di cui calcolare il fattoriale e restituirne il valore corretto
\item Maturare una valutazione critica sull'efficacia dei due approcci
\end{itemize}