\section{Materiali e strumenti}
Linguaggio scelto: \href{https://go.dev/}{Go}

Go, o Golang, è un linguaggio di programmazione relativamente recente (annunciato nel 2009, con la prima release nel 2012) creato da Google. Ecco alcune caratteristiche del linguaggio che mi hanno portato a sceglierlo:

\begin{itemize}
    \item Il linguaggio è staticamente tipizzato (come C, in generale Go è molto influenzato da quest'ultimo), usare un linguaggio di questo tipo aiuta lo studente a capire meglio la differenza tra i vari tipi primitivi.
    \item Ha una sintassi che cerca di mantenere il codice il più leggibile possibile, tendenzialmente facile da ricordare.
    \item Il linguaggio non è orientato agli oggetti, quindi ha una logica più semplice rispetto ad altri linguaggi (tipo Java, che è strettamente OOP).
    \item Il linguaggio è cross-platform.
    \item La documentazione è molto completa.
    \item Il compilatore è molto verboso, molto utile per il debug.
\end{itemize}

Alcune misconception che potrebbero essere causate dalla struttura del linguaggio:

\begin{itemize}
    \item Keyword \texttt{while} assente, al suo posto si usa l'istruzione \texttt{for} con la classica condizione di un \texttt{while}.
    \item La struttura del linguaggio porta a dividere il codice in \textit{package}, all'inizio può risultare complicato capire come vengono gestite.
\end{itemize}