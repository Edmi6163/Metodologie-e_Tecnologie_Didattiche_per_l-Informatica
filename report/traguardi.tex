\section{Traguardi e obiettivi}
Gli studenti al termine di questa attività didattica dovrebbero apprendere le principali differenze tra un'interfaccia costruita basandosi sulla linea di comando o su un'interfaccia grafica, valutando anche diverse opzioni, più o meno intuitive, per lo sviluppo grafico. L'obiettivo è evidenziare le differenze tra i due approcci, formando gli strumenti per valutare quale dei due sia più efficace per una particolare applicazione. 
Si propongono due esempi:
\begin{itemize}
\item Problema semplice, in cui è evidente il beneficio in termini di rapidità di implementazione di usare una CLI
\item Problema più complesso che evidenzia i vantaggi dell'uso di una GUI (attività unplugged)
\end{itemize}